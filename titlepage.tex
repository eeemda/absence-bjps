%% Name, Title, Affiliation, and Contact Information:
\def\myaffiliation{Brown University}
\def\myauthor{Emmerich Davies}
\def\myemail{\small{\href{mailto:emmerich_davies@brown.edu}{emmerich\_davies@brown.edu}}}
\def\mytitle{Absence: Electoral Cycles and Teacher Absenteeism in India}
\def\mykeywords{Emmerich, Davies, Absence, Bureaucracy, India, Development, Education, Teachers, Political Economy}

\hypersetup{
    colorlinks,
    linkcolor={red!80!black},
    citecolor={blue!50!black},
    urlcolor={blue!80!black},
    pdftitle={\mytitle},
    pdfauthor={\myauthor},
    pdfkeywords={\mykeywords}
}

\date{November 5, 2024\\[12pt]\textbf{Word Count: 9,587}}

\begin{document}

\pagenumbering{gobble}

\chapterstyle{article-4}

\pagestyle{kjh}

\title{\bigskip \bigskip {{\mytitle}}}

\published{Conditionally accepted at \emph{British Journal of Political Science}}

\author{\myauthor \\
  \emph{\myaffiliation} \\ \myemail}

\maketitle

\thispagestyle{kjh}

\begin{abstract}
	\footnotesize Public sector worker absence has been commonly cited as a reason for the poor performance of developing country public services. This absenteeism is expensive: a nationally representative sample of villages across India finds that teacher absenteeism costs \$1.5 billion a year. Despite high profile academic, policy, and popular attention to the problem, absenteeism continues to be a chronic problem.  This paper argues that one explanation for absenteeism is the differential attention politicians pay to public services over the cycle of their tenure. Using the case of teachers in India, teachers and politicians are embedded in a dynamic principal-agent relationship that allows for teacher absenteeism when electoral incentives are not salient and results in increased accountability when they are.  I construct a panel of all government schools across India between 2006 and 2018, employ an event study design, and find that teacher absenteeism decreases substantially in the year before an election and is higher in the year after an election. When the number of potential principals increases, absenteeism also increases.  Placebo tests on private school absenteeism find smaller and inconsistent effects of election years on absenteeism in the private sector, lending support for a channel of political control of the bureaucracy around elections.  Political interference in the bureaucracy has a strong effect on bureaucratic performance, and relationships between public sector workers and the state can ameliorate chronic absenteeism.
\end{abstract}

\textbf{Keywords:} accountability, education, electoral cycles, India, political economy, teachers

\clearpage
\pagenumbering{arabic}
\setcounter{page}{1}